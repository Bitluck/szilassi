\documentclass[11pt]{article}
% \documentclass[preview,border=24pt]{standalone}
\usepackage[T2A]{fontenc}
\usepackage[utf8x]{inputenc}
\usepackage[english, russian]{babel}
\usepackage{indentfirst}
\usepackage{array}
\usepackage{fixltx2e}
\usepackage{mathtools}
\usepackage{amssymb}
\usepackage{graphicx}
\usepackage{colortbl}
\usepackage{booktabs}
\graphicspath{ {./} }

\title{\textbf{Многогранник Силаши}}
\date{}
\begin{document}

\maketitle

\textbf{Теорема}. Кроме тетраэдра и многогранника Силаши не существует такого многогранника, что каждая из его граней имеет общее ребро с каждой другой гранью.\\

\emph{Доказательство}. Сначала нужно определить какой вид могут иметь такой многогранник. Выпишем формулу Эйлера для многогранников:
\begin{equation}
  \textnormal{В} - \textnormal{Р} + \textnormal{Г} = 2 - 2g,
  \label{eq:euler}
\end{equation}
где В - количество вершин, Р - количество ребер, Г - количество граней, g - род поверхности.\\

У многогранника будет Г граней, которые будут являтся Г-1 многоугольники (из формулы имеем Г граней, при этом из-за условия, что каждая грань должна иметь общее ребро с другими гранями следует, что они будут Г-1 многоугольниками). При этом будет сходиться ровно три грани в каждой вершине. Так как если будут сходится больше, чем 3, то возникнет ситуация, когда будут две грани не имеющие общего ребра. Если по 2, то будут возникать двуугольники.
Из этого следует, что количество вершин равняется
\begin{equation}
  \textnormal{В} = \frac{\textnormal{Г}(\textnormal{Г}-1)}{3}
\end{equation}
Исходя из того, что все многоугольники попарно имеют общее ребро, то количество ребер будет равно
\begin{equation}
  \textnormal{Р} = \frac{\textnormal{Г}(\textnormal{Г}-1)}{2}
\end{equation}
Подставим в исходную формулу (\ref{eq:euler}), полученные выражения для вершин и рёбер:
\begin{equation}
  \frac{\textnormal{Г}(\textnormal{Г}-1)}{3} + \frac{\textnormal{Г}(\textnormal{Г}-1)}{2} + \textnormal{Г} = 2 - 2g
\end{equation}
После упрощения получаем:
\begin{equation}
  \textnormal{Г}(\textnormal{Г} - 7) = 12(g - 1)
\end{equation}
Выразим через $g$:
\begin{equation}
  g = 1 + \frac{\textnormal{Г}(\textnormal{Г} - 7)}{12}
\end{equation}

Так как $g$ должно быть целым, то $\textnormal{Г} \equiv 0, 3, 4, 7 \pmod{12}$.
При $\textnormal{Г} = 3$ многогранник не реализуем, при $\textnormal{Г} = 4$ - тетраэдр, при $\textnormal{Г} = 7$ - многогранник Силаши. Следующее подходящее $\textnormal{Г} = 12$. Это должен быть многогранник с 44-я вершинами, 66-ю ребрами, 12-ю 11-угольными гранями и 6-ю дырами.

Искомый тип многогранников задается графом, который называется кубический \emph{граф-клетка}\footnote{http://mathworld.wolfram.com/CageGraph.html}. Это регулярный граф с параметрами $(v, g)$ при этом с минимальным количеством вершин, где $v$ - количество ребер графа сходящихся в одной вершине, а $g$ - обхват графа.\\\\
Необходимые определения:

\emph{Регулярный граф} − это граф степени вершин которого равны (в каждую вершину имеет одинаковое количество соседних вершин).

\emph{Кубический граф} − граф в каждой вершине которого сходится ровно три ребра, в нашем случае графы будут $(3, g)-\textnormal{клетки}$.

\emph{Обхват графа} − длина наименьшего цикла в графе (в данном случае количество ребер многоугольника, т.е. $\textnormal{Г}-1$).
\\

Например, при $\textnormal{Г} = 4$ графом будет \emph{$(3, 3)-\textnormal{клетка}$}; при $\textnormal{Г} = 7$ будет \emph{$(3, 6)-\textnormal{клетка}$}\footnote{http://mathworld.wolfram.com/HeawoodGraph.html}. Следующему многограннику соответствует граф $(3, 11)$ с 44-я вершинами. Но минимальное количество вершин для графа \emph{$(3, 11)$}\footnote{http://mathworld.wolfram.com/Balaban11-Cage.html} равняется 112, и это единственный граф с параметрами $(3, 11)$. Откуда следует, что искомого многоранника при Г = 12 не существует.\\

Кроме того известно сколько вершин должно быть у многогранника при определенном $\textnormal{Г}$ и нижнюю оценку на количество вершин в $(3, g)$ графах
\footnote{http://oeis.org/A027383}
\footnote{https://en.wikipedia.org/wiki/Cage\_(graph\_theory)}.

Требуемое количество вершин у искомого многогранника:
\begin{equation}
  \textnormal{В} = \frac{\textnormal{Г}(\textnormal{Г}-1)}{3}
\end{equation}

Нижняя оценка вершин в графе $(v, g)$

\begin{equation}
  \textnormal{В}_{lower} =
  \begin{cases}
    1 + v\sum_{i = 0}^{\frac{g-3}{2}} (v - 1) ^ i, & \text{при нечетном g}\\
    2\sum_{i = 0}^{\frac{g-2}{2}} (v - 1) ^ i, & \text{при четном g}\\
  \end{cases}
\end{equation}

Соответственно для графа $(3, \textnormal{Г}-1)$ нижняя оценка равна
\begin{equation}
  \textnormal{В}_{lower} =
  \begin{cases}
    1 + 3\sum_{i = 0}^{\frac{\text{Г-4}}{2}} 2 ^ i, & \text{при нечетном Г-1}\\
    2\sum_{i = 0}^{\frac{\text{Г-3}}{2}} 2 ^ i, & \text{при четном Г-1}\\
  \end{cases}
\end{equation}

Нижняя оценка растет быстрее, чем требуемое количество вершин в многограннике, вследствие чего для $\textnormal{Г} > 7$ не сущестует графа $(3, $ \textnormal{Г}-1$)$ с необходимым количеством вершин.

\begin{center}
  \begin{tabular}{| c | c | c |}
    \hline
    \textnormal{Г} & \textnormal{B} & $\textnormal{В}_{lower}$ \\ [0.5ex]
    \hline\hline
    4 & 4 & 4 \\
    \hline
    7 & 14 & 14 \\
    \arrayrulecolor{red}\hline
    12 & 44 & 94 \\
    \arrayrulecolor{black}\hline
    15 & 70 & 254 \\
    \hline
    16 & 80 & 382 \\
    \hline
    19 & 114 & 1022 \\
    \hline
    24 & 184 & 6142 \\
    \hline
    27 & 234 & 16382 \\
    \hline
    28 & 252 & 24574 \\
    \hline
    ... & ... & ... \\
    \hline
  \end{tabular}
\end{center}

Из чего можно сделать вывод, что подобных многогранников при Г > 7 не существует. □

\end{document}
